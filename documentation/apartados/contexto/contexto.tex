Este proyecto implementa un sistema de gestión para la empresa cinematográfica ficticia \textit{ElmoCines}. Al tratarse de la gestión de un cine, se habilitarán todas aquellas funcionalidades que permitan simular un uso real de una plataforma de estas características.

\section{Requisitos del sistema}
Antes de comenzar con la creación de la base de datos, debemos plantear cuáles son los requisitos y necesidades que debemos cubrir en un sistema de gestión de cine.
\subsection{Requisitos funcionales}
En este apartado, se detallan los requisitos funcionales que guían el diseño y la implementación de la base de datos para describir las funciones y acciones que la base de datos debe ser capaz de llevar a cabo para satisfacer las necesidades del sistema y de los usuarios:

\begin{itemize}
    \item \textbf{Gestión de clientes:} La base de datos debe ser capaz de soportar operaciones de inserción, recuperación, actualización y borrado (CRUD) sobre los datos de los clientes por el administrador del sitio web. También existirá la función que nos devuelva la película más exitosa.

    \item \textbf{Gestión de películas:} Se deberán poder mostrar todas las películas disponibles en cartelera al usuario, así como los detalles individuales de cada una de ellas al hacer click en su carátula.

    Además, deberán existir funciones que nos permitan eliminar todas las películas, eliminarlas individualmente por su ``id'' e insertarlas.

    \item \textbf{Gestión de reservas:} Se podrán realizar inserciones y borrados de reservas, además de calcular el importe total antes de su realización. Adicionalmente, existirá la posibilidad de obtener el menú más pedido por los clientes.

    \item \textbf{Gestión de sesiones:} Las sesiones podrán ser eliminadas por su id, por el ID de una película (borrando todas las sesiones asociadas a esta) o en su totalidad.

    Por otro lado, se podrá calcular el número de butacas libres de una sesión en específico, listar aquellas con butacas libres y listar los detalles.

    También se podrá buscar una sesión por película, fecha y hora.
\end{itemize}

\subsection{Requisitos no funcionales}
Los requisitos no funcionales definen atributos de calidad que la base de datos debe cumplir para satisfacer las necesidades de los usuarios y garantizar su eficacia y eficiencia:

\begin{itemize}
    \item \textbf{Seguridad:} Los usuarios no serán capaces de acceder a la gestión de los propios usuarios. De este modo, no se podrán realizar operaciones CRUD no deseadas sobre la base de datos.

    \item \textbf{Mantenibilidad:} La base de datos debe ser fácil de mantener y administrar, existiendo una documentación clara de los procedimientos y las tareas necesarias para llevar a cabo dicho mantenimiento.

    \item \textbf{Fiabilidad:} La base de datos será confiable y consistente, por lo que los datos se almacenarán y se recuperarán de forma precisa y sin dar lugar a errores, ya que todos estos estarán debidamente controlados con mecanismos implementados por el desarrollador. Por ello, se garantiza la integridad de los datos y la estabilidad del sistema.

    \item \textbf{Rendimiento:} Se deben procesar consultas de la manera más eficiente posible para así obtener unos tiempos de respuesta óptimos. De esta forma, se maximizará el rendimiento del sistema.
\end{itemize}